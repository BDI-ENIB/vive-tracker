\chapter*{Introduction}
\addcontentsline{toc}{chapter}{Introduction}

En 2017, l'ARENIB (devenu BDI) a présenté un robot innovant, ClArTi à la coupe de France de robotique. Il consistait en une base holonome à 3 roues suédoises surmonté d'un bras multifonction 3 axes et d'un magasin d'éléments de jeux. Pour se positionner, ce robot disposait d'un unique système de triangulation à base d'ondes radios Ultra-Wide Band fourni par notre partenaire Pozyx. Lors des tests à l'association, ce système se révélait être capricieux et peu précis (+/- 10 cm) pour réaliser un asservissement, ce qui a entrainé l'implémentation d'un filtre de Kalman, sans grande amélioration sur les mesures.

Arrivé à la coupe, nous n'avons pas été en mesure de communiquer avec les balises, même au stand : l'environnement radio de la coupe est trés encombré (il n'y a qu'a voir le nombre de réseau Wifi et de téléphones en partage de connexion) et des balises nous ont lâchés. Nous n'avons pas pu être homologué avec ce robot. Nous avons dû fusionner avec l'équipe de Metz, le CRENIM, pour pouvoir participer. La collaboration a marché, leur robot a pu participé au match. Notre robot et notre histoire a attiré l'attention des membres de Planètes sciences et des entreprises partenaires et nous a donc permis de gagner le prix du Jury (!) et de participer à Eurobot. Cette collaboration a perduré et il a été décidé de présenter une équipe commune en 2018 : l'équipe ENIgma. Le CRENIM conçoit la mécanique et l'ARENIB (devenu BDI) conçoit l'électronique et programme les robots.

La création de l'équipe ENIgma a permis de libérer des ressources pour des projets annexes comme la conception d'un système de positionnement par balises, afin de permettre, notamment, le retour de ClArTi à la coupe, un jour. Il a donc été décidé qu'une personne (Evan) développe le système de balises basé sur la technologie Lighthouse, prévu à l'origine pour être utilisé en tandem avec le Pozyx sur ClArTi.

PS : Notez la ressemblance de ClArTi avec GlaDOS. C'est voulu, le thème de la coupe 2017 étant "`Moon Village"' et les créateurs de ClArTi adorant le jeu Portal, il a donc été décidé de réaliser un robot sur ce thème.
