\begin{abstract}

Ce document présente le système de positionnement Vive construit au BDI pour la coupe de France de robotique 2018. Il a pour visée d'aider toute personne intéressée par un système de positionnement à l'échelle d'une table (ou d'une pièce) à comprendre le système Vive de façon à pouvoir l'utiliser ou à pouvoir l'adapter à ses besoins.

La version présentée ici est restreinte à une utilisation en coupe de France de robotique sur un robot de type Summerbot, ceci faute de temps. En revanche, un certain nombre de fonctionnalités ont été envisagés et prévus dans le design des cartes : interpolation avec une centrale inertielle entre deux mesures Vive ou en cas d'occlusion, utilisation via le port USB, communication entre trackers pour récupérer la position de l'adversaire via l'utilisation de modules XBee.

\end{abstract}
