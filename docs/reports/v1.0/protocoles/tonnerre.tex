\section{Tonnerre}

Tonnerre est le protocole de communication entre balises VIVE.
Sa transmission se fait en sans-fil, via les xBee présents sur les balises.

\begin{description}
	\item[1 byte] N° de commande
	\item[1 byte] ID de message, incrémenté à chaque envoi de message. Boucle en cas d'overflow.
	\item[24 bytes] Argument de la commande
\end{description}

Les commandes sont décrites dans le tableau \ref{fig:commandes-tonnerre} et les sous-sections qui suivent.

\begin{table}[htb]
	\centering
	\begin{tabular}{|c|c|l|l|c|l|}
		\hline
		N° & Type  & Description du type & Longueur d'argument & ACK requis & Section \\
		\hline
		0  & ACK   & Acknowledgement     &          & No & \ref{subsec:t:ACK} \\
		\hline
		1  & POS   & Position            & 4 bytes  & No & \ref{subsec:t:POS} \\
		2  & SBPOS & Set Balise Position & 24 bytes & Yes & \ref{subsec:t:SBPOS} \\
		\hline
		3  & PING  & Ping                & 1 byte   & No & \ref{subsec:t:PING-PONG} \\
		4  & PONG  & Ping answer         & 1 byte   & No & \ref{subsec:t:PING-PONG} \\
		\hline
		5  & SBBOX & Set Bounding Box    & 2 bytes  & Yes & \ref{subsec:t:SBBOX} \\
		\hline
		10 & SP    & Start Protocol      & 4 bytes  & Yes & \ref{subsec:t:SP-EP} \\
		11 & EP    & End Protocol        & 4 bytes  & Yes & \ref{subsec:t:SP-EP} \\
		\hline
		20 & SM    & Start Match         &          & Yes & \ref{subsec:t:SM-EM} \\
		21 & EM    & End Match           &          & Yes & \ref{subsec:t:SM-EM} \\
		\hline
		30 & MSG   & Message             & 8 bytes  & Yes & \ref{subsec:t:MSG} \\
		\hline
		69 & QT    & Qute                & 4 bytes  & No & \ref{subsec:t:QT} \\
		\hline
	\end{tabular}
	\caption{Commandes Tonnerre}
	\label{fig:commandes-tonnerre}
\end{table}

\subsection{ACK}
	\label{subsec:t:ACK}

	Accusé de réception envoyé par une balise lorsqu'elle reçoit un quelconque message qui requiert un ACK.
	L'ID de message de l'accusé de réception est le même que celui du message reçu.

\subsection{POS}
	\label{subsec:t:POS}

	Information de position de la balise.

	Composé des coordonnées $x$, $y$, $\theta$ de la balise, réparties comme suit:

	\begin{description}
		\item[$x$] 11 bits
		\item[$y$] 12 bits
		\item[$\theta$] 9 bits
	\end{description}

\subsection{SBPOS}
	\label{subsec:t:SBPOS}

	Information de position du phare.

	Composé des coordonnées $x$, $y$, $z$, $\alpha$, $\beta$, $\gamma$ de la balise, au format \verb|float|.

	Cette commande est utilisée au début du match pour informer les balises de la position exacte du phare par rapport au $0$ de la table.

	\emph{Si les informations envoyées par cette commande ne sont pas correctement reçues, la position renvoyée par la balise sera erronée.}

\subsection{PING--PONG}
	\label{subsec:t:PING-PONG}

	Les commandes PING et PONG sont liées, PONG est l'accusé de réception de PING.

	Elles servent à vérifier que la communication entre deux balises s'effectue correctement.

	Leur argument est un octet généré aléatoirement lors du PING et renvoyé exactement tel quel par le PONG.

	L'ID de message du PONG est celui du PING correspondant.

\subsection{SBBOX}
	\label{subsec:t:SBBOX}

	Définit le diamètre du robot cible.

	Son argument est un \verb|float|, le diamètre en question.

\subsection{SP--EP}
	\label{subsec:t:SP-EP}

	Commandes informant les autres robots du début (de la fin) d'un protocole.

	L'argument est l' n° de protocole sous la forme d'un entier signé 32 bits.

\subsection{SM--EM}
	\label{subsec:t:SM-EM}

	Commandes informant les autres robots du début (de la fin) du match.

\subsection{MSG}
	\label{subsec:t:MSG}

	Commande de message générique de longueur 8 octets.

\subsection{QT}
	\label{subsec:t:QT}

	Commande de contrôle de l'affichage LED WS2812 de la couronne.

	Utilise le protocole Qute décrit en section \ref{sec:Qute}.
