\section{Arsenal}

Arsenal est le protocole de communication entre une balise et le microcontrôleur
du robot portant cette balise.

Le transport se fait par liaison série UART ou USB. C'est un protocole texte, lisible par un humain.

Les commandes sont composées de plusieurs parties, séparées par une espace standard (\verb|U+0020|).
Les commandes sont séparées entre elles par un point-virgule (\verb|;|).

Une commande est composée des partie suivantes:

\begin{description}
	\item[Type] Type de commande
	\item[ID VIVE] Identifiant VIVE
	\item[Arguments] Données utiles du message
\end{description}

Les commandes sont décrites dans le tableau \ref{fig:commandes-arsenal} et les sous-sections qui suivent.

L'ID VIVE est un identifiant assigné à chaque balise avant les matchs. Il est défini dans les messages comme décrit dans le tableau \ref{fig:a:sens-id-vive}.

\begin{table}[htb]
	\centering
	\begin{tabular}{|r|c|c|} \hline
		\diagbox{Présence d'ID}{Provenance} & Microcontrôleur & Balise        \\ \hline
		Oui                                 & Balise cible    & Balise source \\ \hline
		Non                                 & Broadcast       & X             \\ \hline
	\end{tabular}
	\caption{Sens de l'ID VIVE}
	\label{fig:a:sens-id-vive}
\end{table}


\begin{table}[htb]
	\centering
	\begin{tabular}{|c|l|l|c|l|}
		\hline
		Type        & Description du type & Longueur d'argument & ID requis & Section            \\
		\hline
		POS         & Position            & 4 bytes             & No        & \ref{subsec:a:POS}   \\
		\hline
		SBPOS       & Set Balise Position & 24 bytes            & Yes       & \ref{subsec:a:SBPOS} \\
		SBBOX       & Set Bounding Box    & 2 bytes             & Yes       & \ref{subsec:a:SBBOX} \\
		SETID       & Set VIVE ID         &                     &           & \ref{subsec:a:SBBOX} \\
		\hline
		STPROTOCOL  & Start Protocol      & 4 bytes             & Yes       & \ref{subsec:a:SP-EP} \\
		ENDPROTOCOL & End Protocol        & 4 bytes             & Yes       & \ref{subsec:a:SP-EP} \\
		\hline
		STMATCH     & Start Match         &                     & Yes       & \ref{subsec:a:SM-EM} \\
		ENDMATCH    & End Match           &                     & Yes       & \ref{subsec:a:SM-EM} \\
		\hline
		MSG         & Message             & 8 bytes             & Yes       & \ref{subsec:a:MSG}   \\
		\hline
		QT          & Qute                & 4 bytes             & No        & \ref{subsec:a:QT}    \\
		\hline
	\end{tabular}
	\caption{Commandes Arsenal}
	\label{fig:commandes-arsenal}
\end{table}


\subsection{POS}
	\label{subsec:a:POS}

	Information de position de la balise, envoyé par la balise au microcontrôleur uniquement.

	Composé des coordonnées $x$, $y$, $\theta$ de la balise ID VIVE, réparties comme suit:

	\begin{description}
		\item[$x$] 11 bits
		\item[$y$] 12 bits
		\item[$\theta$] 9 bits
	\end{description}

\subsection{SBPOS}
	\label{subsec:a:SBPOS}

	Information de position du phare.

	Composé des coordonnées $x$, $y$, $z$, $\alpha$, $\beta$, $\gamma$ du phare, au format \verb|float|.

	Cette commande est utilisée au début du match pour informer les balises de la position exacte du phare par rapport au $0$ de la table. Le robot principal utilise cette commande qui déclenche ensuite un broadcast de la commande Tonnerre correspondante, cf section \ref{subsec:t:SBPOS}.

	\emph{Si les informations envoyées par cette commande ne sont pas correctement reçues, la position renvoyée par la balise sera erronée.}

\subsection{SBBOX}
	\label{subsec:a:SBBOX}

	Définit le diamètre du robot cible.

	Son argument est un \verb|float|, le diamètre en question.

	Envoyé par le robot principal aux balises placées sur les robots adverses pour leur permettre de signaler une collision imminente.

\subsection{STPROTOCOL--ENDPROTOCOL}
	\label{subsec:a:SP-EP}

	Commandes informant les autres robots du début (de la fin) d'un protocole.

	L'argument est l' n° de protocole sous la forme d'un entier signé 32 bits.

	Lors de leur émission, l'ID VIVE n'est pas requis, il sera défini par le VIVE récepteur en fonction de la provenance.

\subsection{STMATCH--ENDMATCH}
	\label{subsec:a:SM-EM}

	Commandes informant les autres robots du début (de la fin) du match.

	Lors de leur émission, l'ID VIVE n'est pas requis, il sera défini par le VIVE récepteur en fonction de la provenance.

\subsection{MSG}
	\label{subsec:a:MSG}

	Commande de message générique de longueur 8 octets.

	Si un ID VIVE est défini lors de l'envoi, c'est uniquement le robot correspondant qui recevra le message. Sinon, c'est un broadcast.

\subsection{QT}
	\label{subsec:a:QT}

	Commande de contrôle de l'affichage LED WS2812 de la couronne.

	Utilise le protocole Qute décrit en section \ref{sec:Qute}.

	Si un ID VIVE est défini lors de l'envoi, c'est uniquement le robot correspondant qui recevra le message. Sinon, c'est un broadcast.
