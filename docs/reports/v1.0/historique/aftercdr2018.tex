
\section{Retour après la coupe 2018}

Lors de la coupe 2018, la fabrication des balises n'étant pas terminé, nous avons donc consacré une bonne partie du temps de la coupe à construire au moins 2 balises (une pour chaque robot). Ce faisant, la qualité de l'assemblage réalisé est mauvaise ce qui a conduit a casser certaines cartes IRInterfaces. Le composant Triad, très petit et soudé par le dessous avec ses billes de soudure, est très fragile : le moindre frottement de la carte contre un autre objet (ici, la coque de la balise) peut endommager voire chasser la puce. Nous sommes donc obligé de remplacer la carte par une autre. Nous en avions un certain nombre en réserve mais vu la fragilité du composant et le nombre de cartes cassés à la coupe, il serait bien de revoir le design des balises pour avoir une fixation des cartes IRInterfaces plus propres.

De plus, les balises étant conçues pour être très compactes, la marge de manœuvre pour câbler les IRInterfaces aux IRoises était très petite. Nous avons dû raccourcir des fils femelles-femelles pour breadboard. Le câblage devenait illisible, bordélique et empêchait la fermeture des deux parties de la balise, ce qui nous a poussé à découper à la Dremel des espaces pour pouvoir les refermer. De ce câblage, nous avons eu des problèmes de communication entre le PSoC et les Triad, ce qui a poussé à désactiver certaines LEDs dans le code (d'où les \verb|+=2| dans les boucles du \verb|Position_finder| et de \verb|VIVE_sensors|).

Cela était sans compter sur une erreur des organisateurs sur la fabrication des mâts centraux de balises. La CAO de la table indiquait un mât de balise plus large que les schémas fournis avec le règlement. Arrivé à la coupe, les mâts des tables ont été construits sur le modèle CAO et non sur le modèle du règlement (qui est la version qui fait foi). Nous avons dû ajuster notre fixation, pour se rendre compte, à l'homologation, que nous avions raison au départ et qu'il fallait réajuster notre support dans l'autre sens !

Hormis tout ces problèmes (ne mettant pas en doute la précision et la fiabilité du système électronique), nous avons pu tester, sur le stand, que la technologie fonctionnait parfaitement. Comme indiqué ci-dessus, la finalisation des balises à la coupe ne nous a pas permis de tester (et d'utiliser) le système avec le robot, en condition réelle. Cependant, nous sommes largement confiant sur la fiabilité de la technologie Lighthouse pour garantir les performances lors des prochaines coupes.
